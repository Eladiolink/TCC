
\chapter{Resultados}\label{resultados}

No capítulo atual, é apresentado tabelas com os dados coletados, os dados abrangem o tamanho do binário emitido pelos compiladores e quantidade de memória utilizada ao executa-los. Ademais, é apresentado estatísticas descritivas sobre o tempo de execução coletado. Por fim, nesse capítulo é analisado os dados apresentados visando responder as questões de pesquisa formuladas no capítulo inicial.

\section{Análise do tempo de execução}

A análise do tempo de execução será realizada de forma semelhante a análise feita na seção anterior, isto é, para cada tripla(algoritmo, navegador e tamanho de entrada) será calculado a razão entre o tempo de execução apresentado pelo Cheerp sobre o tempo apresentado pelo seu rival.

\begin{longtable}{lrr}\caption{Razões do tempo de execução \label{time_stat_analisys}} \\
\toprule
{} &  Tempo Exec. (entrada grande) &  Tempo Exec. (entrada média) \\
\midrule
\endfirsthead

\toprule
{} &  Tempo Exec. (entrada grande) &  Tempo Exec. (entrada média) \\
\midrule
\endhead
\midrule
\multicolumn{3}{r}{{Continued on next page}} \\
\midrule
\endfoot

\bottomrule
\endlastfoot
Média      &                         1.578 &                        2.095 \\
Desvio p.  &                         0.637 &                        1.009 \\
Min.       &                         0.978 &                        0.778 \\
1° quartil &                         1.043 &                        1.199 \\
2° quartil &                         1.312 &                        2.064 \\
3° quartil &                         1.859 &                        2.915 \\
Max.       &                         3.180 &                        5.000 \\
\end{longtable}


\section{Questões de Pesquisa}

Diante da análise realiza, há informações necessárias para responder as questões de pesquisa enunciadas no capítulo inicial. Portanto, a seguir será respondido cada uma delas, utilizando as conclusões obtidas nesse capítulo.

\begin{description}
    \item[QP01] Qual dos dois compiladores estudados emite um binário com tamanho menor?

    Independente do tamanho da entrada, na média o Cheerp apresentou um binário 10\% menor que o binário emitido pelo Emscripten. Ademais, a variação desse percentual foi muito pequena, logo, em todos os algoritmos utilizados esse resultado se mostrou verdadeiro.

    \item[QP02] Entre os dois, qual produz um binário que utiliza menos memória, considerando o tamanho inicial da memória igual para ambos?
\end{description}

\subsection{Temperatura 0}

\begin{table}[H]
\centering
\caption{Taxa de acertos geral por modelo}
\label{tab:acuracia_geral}
\begin{tabular}{l c}
\toprule
Modelo & Acurácia \\
\midrule
GPT & 0.83\% \\
Gemini & 0.79\% \\
Deepseek & 0.82\% \\
\bottomrule
\end{tabular}
\end{table}


\begin{table}[H]
\centering
\caption{Taxa de acertos por ano}
\label{tab:acuracia_ano}
\begin{tabular}{l c c c}
\toprule
 & GPT & Gemini & Deepseek \\
Ano &  &  &  \\
\midrule
2014 & 0.76\% & 0.72\% & 0.79\% \\
2017 & 0.85\% & 0.95\% & 0.90\% \\
2021 & 0.91\% & 0.74\% & 0.78\% \\
\bottomrule
\end{tabular}
\end{table}


\begin{table}[H]
\centering
\caption{Taxa de acertos por área de conhecimento}
\label{tab:acuracia_area}
\begin{tabular}{l c c c}
\toprule
 & GPT & Gemini & Deepseek \\
Área do Conhecimento &  &  &  \\
\midrule
Análise de Algoritmos & 1.00\% & 0.83\% & 1.00\% \\
Arquitetura e Organização de Computadores & 1.00\% & 0.00\% & 1.00\% \\
Banco de Dados & 1.00\% & 1.00\% & 1.00\% \\
Conhecimento Geral & 1.00\% & 0.93\% & 0.87\% \\
Construção de Compiladores & 1.00\% & 1.00\% & 1.00\% \\
Engenharia de Software & 0.89\% & 0.89\% & 0.78\% \\
Estrutura de Dados & 0.75\% & 0.62\% & 0.75\% \\
Fundamentos Matemáticos & 0.60\% & 0.80\% & 0.80\% \\
Inteligencia Artificial & 1.00\% & 1.00\% & 1.00\% \\
Redes de Computadores & 1.00\% & 1.00\% & 1.00\% \\
Segurança da Informação & 0.67\% & 1.00\% & 1.00\% \\
Sistemas Distribuídos & 1.00\% & 1.00\% & 1.00\% \\
Sistemas Operacionais & 0.43\% & 0.29\% & 0.43\% \\
Teoria da Computação & 0.78\% & 0.78\% & 0.78\% \\
\bottomrule
\end{tabular}
\end{table}


\subsection{Temperatura 0.4}

\begin{table}[H]
\centering
\caption{Taxa de acertos geral por modelo}
\label{tab:acuracia_geral}
\begin{tabular}{l c}
\toprule
Modelo & Acurácia \\
\midrule
GPT & 0.83\% \\
Gemini & 0.81\% \\
Deepseek & 0.82\% \\
\bottomrule
\end{tabular}
\end{table}

\begin{table}[H]
\centering
\caption{Taxa de acertos por ano}
\label{tab:acuracia_ano}
\begin{tabular}{l c c c}
\toprule
 & GPT & Gemini & Deepseek \\
Ano &  &  &  \\
\midrule
2014 & 0.76\% & 0.72\% & 0.79\% \\
2017 & 0.85\% & 0.95\% & 0.90\% \\
2021 & 0.91\% & 0.78\% & 0.78\% \\
\bottomrule
\end{tabular}
\end{table}

\begin{table}[H]
\centering
\caption{Taxa de acertos por área de conhecimento}
\label{tab:acuracia_area}
\begin{tabular}{l c c c}
\toprule
 & GPT & Gemini & Deepseek \\
Área do Conhecimento &  &  &  \\
\midrule
Análise de Algoritmos & 1.00\% & 0.83\% & 1.00\% \\
Arquitetura e Organização de Computadores & 1.00\% & 1.00\% & 1.00\% \\
Banco de Dados & 1.00\% & 1.00\% & 1.00\% \\
Conhecimento Geral & 1.00\% & 0.93\% & 0.87\% \\
Construção de Compiladores & 1.00\% & 1.00\% & 1.00\% \\
Engenharia de Software & 0.89\% & 0.89\% & 0.78\% \\
Estrutura de Dados & 0.75\% & 0.62\% & 0.75\% \\
Fundamentos Matemáticos & 0.60\% & 0.80\% & 0.80\% \\
Inteligencia Artificial & 1.00\% & 1.00\% & 1.00\% \\
Redes de Computadores & 1.00\% & 1.00\% & 1.00\% \\
Segurança da Informação & 0.67\% & 1.00\% & 1.00\% \\
Sistemas Distribuídos & 1.00\% & 1.00\% & 1.00\% \\
Sistemas Operacionais & 0.43\% & 0.29\% & 0.43\% \\
Teoria da Computação & 0.78\% & 0.78\% & 0.78\% \\
\bottomrule
\end{tabular}
\end{table}


\subsection{Temperatura 0.8}

\begin{table}[H]
\centering
\caption{Taxa de acertos geral por modelo}
\label{tab:acuracia_geral}
\begin{tabular}{l c}
\toprule
Modelo & Acurácia \\
\midrule
GPT & 0.83\% \\
Gemini & 0.83\% \\
Deepseek & 0.83\% \\
\bottomrule
\end{tabular}
\end{table}

\begin{table}[H]
\centering
\caption{Taxa de acertos por ano}
\label{tab:acuracia_ano}
\begin{tabular}{l c c c}
\toprule
 & GPT & Gemini & Deepseek \\
Ano &  &  &  \\
\midrule
2014 & 0.76\% & 0.76\% & 0.79\% \\
2017 & 0.85\% & 0.95\% & 0.90\% \\
2021 & 0.91\% & 0.83\% & 0.83\% \\
\bottomrule
\end{tabular}
\end{table}

\begin{table}[H]
\centering
\caption{Taxa de acertos por área de conhecimento}
\label{tab:acuracia_area}
\begin{tabular}{l c c c}
\toprule
 & GPT & Gemini & Deepseek \\
Área do Conhecimento &  &  &  \\
\midrule
Análise de Algoritmos & 1.00\% & 0.83\% & 1.00\% \\
Arquitetura e Organização de Computadores & 1.00\% & 1.00\% & 1.00\% \\
Banco de Dados & 1.00\% & 1.00\% & 1.00\% \\
Conhecimento Geral & 1.00\% & 0.93\% & 0.87\% \\
Construção de Compiladores & 1.00\% & 1.00\% & 1.00\% \\
Engenharia de Software & 0.89\% & 0.89\% & 0.78\% \\
Estrutura de Dados & 0.75\% & 0.62\% & 0.75\% \\
Fundamentos Matemáticos & 0.60\% & 0.80\% & 0.80\% \\
Inteligencia Artificial & 1.00\% & 1.00\% & 1.00\% \\
Redes de Computadores & 1.00\% & 1.00\% & 1.00\% \\
Segurança da Informação & 0.67\% & 1.00\% & 1.00\% \\
Sistemas Distribuídos & 1.00\% & 1.00\% & 1.00\% \\
Sistemas Operacionais & 0.43\% & 0.43\% & 0.43\% \\
Teoria da Computação & 0.78\% & 0.89\% & 0.89\% \\
\bottomrule
\end{tabular}
\end{table}


\subsection{Temperatura 1.2}

\begin{table}[H]
\centering
\caption{Taxa de acertos geral por modelo}
\label{tab:acuracia_geral}
\begin{tabular}{l c}
\toprule
Modelo & Acurácia \\
\midrule
GPT & 0.83\% \\
Gemini & 0.83\% \\
Deepseek & 0.82\% \\
\bottomrule
\end{tabular}
\end{table}


\begin{table}[H]
\centering
\caption{Taxa de acertos por ano}
\label{tab:acuracia_ano}
\begin{tabular}{l c c c}
\toprule
 & GPT & Gemini & Deepseek \\
Ano &  &  &  \\
\midrule
2014 & 0.76\% & 0.72\% & 0.79\% \\
2017 & 0.85\% & 1.00\% & 0.90\% \\
2021 & 0.91\% & 0.83\% & 0.78\% \\
\bottomrule
\end{tabular}
\end{table}


\begin{table}[H]
\centering
\caption{Taxa de acertos por área de conhecimento}
\label{tab:acuracia_area}
\begin{tabular}{l c c c}
\toprule
 & GPT & Gemini & Deepseek \\
Área do Conhecimento &  &  &  \\
\midrule
Análise de Algoritmos & 1.00\% & 0.83\% & 1.00\% \\
Arquitetura e Organização de Computadores & 1.00\% & 1.00\% & 1.00\% \\
Banco de Dados & 1.00\% & 1.00\% & 1.00\% \\
Conhecimento Geral & 1.00\% & 0.93\% & 0.87\% \\
Construção de Compiladores & 1.00\% & 1.00\% & 1.00\% \\
Engenharia de Software & 0.89\% & 0.89\% & 0.78\% \\
Estrutura de Dados & 0.75\% & 0.62\% & 0.75\% \\
Fundamentos Matemáticos & 0.60\% & 0.80\% & 0.80\% \\
Inteligencia Artificial & 1.00\% & 1.00\% & 1.00\% \\
Redes de Computadores & 1.00\% & 1.00\% & 1.00\% \\
Segurança da Informação & 0.67\% & 1.00\% & 1.00\% \\
Sistemas Distribuídos & 1.00\% & 1.00\% & 1.00\% \\
Sistemas Operacionais & 0.43\% & 0.57\% & 0.43\% \\
Teoria da Computação & 0.78\% & 0.78\% & 0.78\% \\
\bottomrule
\end{tabular}
\end{table}


\subsection{Temperatura 1.6}

\begin{table}[H]
\centering
\caption{Taxa de acertos geral por modelo}
\label{tab:acuracia_geral}
\begin{tabular}{l c}
\toprule
Modelo & Acurácia \\
\midrule
GPT & 0.83\% \\
Gemini & 0.79\% \\
Deepseek & 0.78\% \\
\bottomrule
\end{tabular}
\end{table}


\begin{table}[H]
\centering
\caption{Taxa de acertos por ano}
\label{tab:acuracia_ano}
\begin{tabular}{l c c c}
\toprule
 & GPT & Gemini & Deepseek \\
Ano &  &  &  \\
\midrule
2014 & 0.76\% & 0.72\% & 0.76\% \\
2017 & 0.85\% & 0.95\% & 0.85\% \\
2021 & 0.91\% & 0.74\% & 0.74\% \\
\bottomrule
\end{tabular}
\end{table}


\begin{table}[H]
\centering
\caption{Taxa de acertos por área de conhecimento}
\label{tab:acuracia_area}
\begin{tabular}{l c c c}
\toprule
 & GPT & Gemini & Deepseek \\
Área do Conhecimento &  &  &  \\
\midrule
Análise de Algoritmos & 1.00\% & 0.83\% & 0.67\% \\
Arquitetura e Organização de Computadores & 1.00\% & 1.00\% & 1.00\% \\
Banco de Dados & 1.00\% & 1.00\% & 1.00\% \\
Conhecimento Geral & 1.00\% & 0.87\% & 0.80\% \\
Construção de Compiladores & 1.00\% & 1.00\% & 1.00\% \\
Engenharia de Software & 0.89\% & 0.89\% & 0.78\% \\
Estrutura de Dados & 0.75\% & 0.62\% & 0.75\% \\
Fundamentos Matemáticos & 0.60\% & 0.80\% & 0.80\% \\
Inteligencia Artificial & 1.00\% & 1.00\% & 1.00\% \\
Redes de Computadores & 1.00\% & 1.00\% & 1.00\% \\
Segurança da Informação & 0.67\% & 1.00\% & 1.00\% \\
Sistemas Distribuídos & 1.00\% & 1.00\% & 1.00\% \\
Sistemas Operacionais & 0.43\% & 0.29\% & 0.43\% \\
Teoria da Computação & 0.78\% & 0.78\% & 0.78\% \\
\bottomrule
\end{tabular}
\end{table}


\subsection{Temperatura 2.0}

\begin{table}[H]
\centering
\caption{Taxa de acertos geral por modelo}
\label{tab:acuracia_geral}
\begin{tabular}{l c}
\toprule
Modelo & Acurácia \\
\midrule
GPT & 0.81\% \\
Gemini & 0.83\% \\
Deepseek & 0.85\% \\
\bottomrule
\end{tabular}
\end{table}


\begin{table}[H]
\centering
\caption{Taxa de acertos por ano}
\label{tab:acuracia_ano}
\begin{tabular}{l c c c}
\toprule
 & GPT & Gemini & Deepseek \\
Ano &  &  &  \\
\midrule
2014 & 0.69\% & 0.76\% & 0.83\% \\
2017 & 0.90\% & 1.00\% & 1.00\% \\
2021 & 0.87\% & 0.78\% & 0.74\% \\
\bottomrule
\end{tabular}
\end{table}


\begin{table}[H]
\centering
\caption{Taxa de acertos por área de conhecimento}
\label{tab:acuracia_area}
\begin{tabular}{l c c c}
\toprule
 & GPT & Gemini & Deepseek \\
Área do Conhecimento &  &  &  \\
\midrule
Análise de Algoritmos & 0.83\% & 0.83\% & 0.83\% \\
Arquitetura e Organização de Computadores & 1.00\% & 1.00\% & 1.00\% \\
Banco de Dados & 1.00\% & 1.00\% & 1.00\% \\
Conhecimento Geral & 0.93\% & 0.93\% & 0.80\% \\
Construção de Compiladores & 1.00\% & 1.00\% & 1.00\% \\
Engenharia de Software & 0.89\% & 0.89\% & 0.78\% \\
Estrutura de Dados & 0.62\% & 0.62\% & 1.00\% \\
Fundamentos Matemáticos & 0.60\% & 0.80\% & 0.80\% \\
Inteligencia Artificial & 1.00\% & 1.00\% & 1.00\% \\
Redes de Computadores & 1.00\% & 1.00\% & 1.00\% \\
Segurança da Informação & 0.67\% & 1.00\% & 1.00\% \\
Sistemas Distribuídos & 1.00\% & 1.00\% & 1.00\% \\
Sistemas Operacionais & 0.43\% & 0.43\% & 0.57\% \\
Teoria da Computação & 0.89\% & 0.89\% & 0.89\% \\
\bottomrule
\end{tabular}
\end{table}
