% ----------------------------------------------------------
\chapter{Introdução}\label{cap1}
% ----------------------------------------------------------

A evolução da Inteligência Artificial (IA) tem transformado paradigmas em diversos setores da sociedade, migrando de sistemas baseados em regras rígidas para modelos capazes de gerar conteúdo e interpretar nuances humanas. Nesse cenário, o Processamento de Linguagem Natural (PLN) ganhou destaque com o surgimento dos Grandes Modelos de Linguagem, do inglês \textit{Large Language Models} (LLM). Esses modelos, fundamentados em arquiteturas \textit{Transformer} e treinados com volumes massivos de dados, superaram as limitações das tecnologias anteriores, demonstrando capacidades avançadas de compreensão, tradução e geração de texto coerente. Tal avanço permitiu que a IA deixasse de ser apenas uma ferramenta de classificação para se tornar um instrumento de apoio cognitivo em tarefas complexas \cite{mohamed2024hands}.

Dentro desse espectro, o Processamento de Linguagem Natural (PLN) consolida-se como uma subárea interdisciplinar, situada na interseção entre a Ciência da Computação, a Inteligência Artificial e a Linguística \cite{Tilton2016Introduction}. O objetivo primordial do PLN é capacitar sistemas computacionais a compreender, interpretar e manipular a linguagem humana de forma significativa, superando a barreira entre a comunicação natural e o código de máquina. Diferentemente das linguagens de programação, que são estruturadas e inequívocas, a linguagem natural é inerentemente complexa, repleta de ambiguidades semânticas, variações sintáticas e contextos implícitos. Historicamente, essa complexidade limitava os sistemas tradicionais, que dependiam de regras manuais rígidas e falhavam em capturar as nuances da comunicação humana real.

A superação dessas barreiras ocorreu com o advento de técnicas de Aprendizado Profundo (\textit{Deep Learning}) e, mais recentemente, com a popularização das LLMs. Fundamentados majoritariamente na arquitetura \textit{Transformer}, esses modelos utilizam mecanismos de atenção para ponderar a relevância de diferentes elementos em uma sequência textual, permitindo a compreensão de dependências de longo prazo e contextos complexos. Ao serem treinados em \textit{datasets} massivos que abrangem grande parte do conhecimento disponível na internet, os LLMs adquiriram uma capacidade de generalização inédita. Isso permitiu que a tecnologia evoluísse de tarefas simples de classificação para a geração autônoma de texto, raciocínio lógico e adaptação a diferentes domínios de conhecimento sem a necessidade de retreinamento específico \cite{zhao2023survey}.

Com todo o avanço nessa área da PLN, surge a possibilidade de investigar formas de aplicação dessa tecnologia, e uma das mais promissoras é o apoio à educação. Trabalhos que antes demandavam tempo e esforço humano podem ser automatizados e aprimorados. Assim, essa tecnologia pode ser explorada para a automatização de avaliações ou correções que se caracterizam por serem detalhados, oportunos e de apoio , aspectos cruciais para o desenvolvimento e aprendizado do aluno \cite{liew2024automated}.

O uso de LLMs em tarefas de processamento de linguagem natural tem se mostrado promissor em diversos contextos, incluindo aplicações na área educacional. Em especial, na área de Computação, muitos processos avaliativos como a correção de questões de múltipla escolha, que frequentemente envolvem conceitos técnicos ainda são realizados manualmente ou com sistemas limitados em flexibilidade e capacidade interpretativa \cite{Das2021Multiple-choice}. Nesse cenário, os LLMs se destacam por oferecerem maior adaptabilidade e por possibilitarem a geração de feedbacks mais contextualizados e pedagógicos. No entanto, sua real eficácia nesse tipo de aplicação ainda carece de investigações mais aprofundadas. 

Estudos recentes têm explorado a competência de LLMs em domínios de conhecimento especializados na área de educação médica\cite{kung2023performance}, Observa-se uma lacuna no que diz respeito ao uso dos LLMs especificamente em questões do Exame Nacional de Desempenho de Estudantes (ENADE) voltadas para cursos da área de Computação. Assim, este trabalho justifica-se pela necessidade de avaliar o desempenho de diferentes LLMs na resolução e interpretação de questões desse exame, buscando compreender suas limitações, potencialidades e possíveis contribuições para o aprimoramento de processos avaliativos na educação superior.

Para validar a eficácia dos modelos nesse nível de exigência, o ENADE apresenta-se como um instrumento de referência ideal. Instituído pelo Ministério da Educação (MEC), o exame transcende a simples verificação de memorização, sendo desenhado para avaliar o desenvolvimento de competências, habilidades e a capacidade de síntese dos graduandos frente a problemas reais da profissão \cite{inep2023}. Utilizar o ENADE como parâmetro de teste para as LLMs é ideal, pois suas questões frequentemente interdisciplinares e contextualizadas impõem um desafio de interpretação superior ao de exercícios convencionais. Portanto, verificar se esses modelos conseguem resolver satisfatoriamente tais questões é um passo decisivo para atestar sua viabilidade como ferramentas de apoio pedagógico no ensino superior de Computação.

Portanto, torna-se evidente que os LLMs representam uma oportunidade relevante para o aprimoramento de processos avaliativos no ensino superior, especialmente em áreas que exigem interpretação técnica e contextual, como a Computação. A escolha do ENADE como objeto de análise possibilita investigar o desempenho desses modelos em um cenário real, complexo e alinhado às competências esperadas de futuros profissionais. Assim, este trabalho propõe avaliar a eficácia de diferentes LLMs na resolução de questões do exame, buscando identificar suas potencialidades, limitações e contribuições para aplicações educacionais. A partir dessa análise, pretende-se oferecer subsídios para a incorporação responsável e eficiente dessas tecnologias em práticas pedagógicas, alinhando-se às diretrizes globais para o uso de IA na educação \cite{holmes2023guidance}, e reforçando seu papel como ferramentas de apoio ao processo de aprendizagem e avaliação.


\section{Questões de Pesquisa}

O desenvolvimento desse trabalho foi elaborado com objetivo de responder as seguintes questões de pesquisa:

\begin{description}
    \item[QP01] O quão acertivo pode ser uma LLM's na resolução de questões em computação?
    \item[QP02] Dentre as LLM's selecionadas para o estudo, qual obteve o melhor resultado na resolução de questões?
    \item[QP03] Alguma das LLM'S se mostra superior nos acertos em alguma área específica da computação?
\end{description}

\section{Objetivos}

Os objetivos deste trabalho são subdivididos em objetivos gerais e objetivos específicos. Estes são:

\subsection{Objetivo Geral}

Avaliar comparativamente o desempenho de grandes modelos de linguagem na resolução de questões objetivas do ENADE aplicadas a cursos da área de Computação, com base no gabarito oficial.

\subsection{Objetivos Específicos}

Os objetivos específicos são:

\begin{itemize}
    \item Realizar uma análise da literatura recente sobre grandes modelos de linguagem com foco em tarefas de resposta a perguntas e compreensão de texto;

    \item Selecionar, com base na literatura analisada, os modelos de linguagem para fins de avaliação comparativa;
    
    \item Coletar e organizar questões objetivas de múltipla escolha do ENADE, aplicadas desde 2004, para os cursos da área de Computação;
    
    \item Submeter as questões selecionadas aos modelos escolhidos, registrando sistematicamente as respostas geradas;

    \item Avaliar e comparar o desempenho dos modelos considerando a granularidade de curso, a partir dos gabaritos oficiais disponibilizados pelo INEP.
\end{itemize}

\section{Organização do Trabalho}

Esse trabalho é organizado como segue: No capítulo \ref{fundamentacao} são apresentados os conceitos base para melhor entendimento das tecnologias abordadas. Portanto, o capítulo apresenta uma introdução sobre os grandes modelos de linguagem (Large Language Models), seguido por apresentar o conceito da tecnologia e uma contextualização histórica sobre a evolução da área da PLN até a chegada dos transformers. Ao final do capítulo é também listado as principais pesquisas relacionadas. No capítulo seguinte, \ref{delineamento}, é descrito os passos necessários para realizar o experimento desejado assim como o ambiente adotado para execução da pesquisa. No capítulo \ref{resultados} é apresentado os dados coletados no experimento executado, em seguida é feito uma análise dos dados visando responder as questões de pesquisa. Por fim, no capítulo \ref{conclusoes} é sintetizado o que foi realizado na pesquisa assim como os resultados obtidos ao final da análise de dados.

