% ----------------------------------------------------------
\chapter{Introdução}\label{cap1}
% ----------------------------------------------------------

O uso de grandes modelos de linguagem, do inglês Large Language Model (LLM), já está se tornando cada vez mais comum na sociedade atual, apoiando diversas áreas em que se necessita auxílio em atividades textuais. Recentemente, o desenvolvimento desses modelos impulsionou avanços significativos na área de Processamento de Linguagem Natural (PLN), criando diversas oportunidades de uso tanto em ambientes profissionais quanto educacionais. Os modelos de LLM atuais avançam para além dos modelos de linguagem tradicionais ao integrar datasets maiores e arquiteturas transformer, destacando-se na aprendizagem a partir de dados extensivos e alcançando resultados de ponta em tarefas de PLN. Isso inclui compreensão da linguagem, geração de linguagem, tradução automática, geração de diálogo, análise de sentimentos e sumarização de conteúdo \cite{mohamed2024hands}.

Com todo o avanço nessa área da PLN, surge a possibilidade de investigar formas de aplicação dessa tecnologia, e uma das mais promissoras é o apoio à educação. Trabalhos que antes demandavam tempo e esforço humano podem ser automatizados e aprimorados. Assim, essa tecnologia pode ser explorada para a automatização de avaliações e geração de feedbacks  que se caracterizam por serem detalhados, oportunos e de apoio , aspectos cruciais para o desenvolvimento e aprendizado do aluno \cite{liew2024automated}.

\section{Questões de Pesquisa}

O desenvolvimento desse trabalho foi elaborado com objetivo de responder as seguintes questões de pesquisa:

\begin{description}
    \item[QP01] O quão acertivo pode ser uma LLM's na resolução de questões em computação?
    \item[QP02] Dentre as LLM's selecionadas para o estudo, qual obteve o melhor resultado na resolução de questões?
    \item[QP03] Alguma das LLM'S se mostra superior nos acertos em alguma área específica da computação?
\end{description}

\section{Objetivos}

Os objetivos deste trabalho são subdivididos em objetivos gerais e objetivos específicos. Estes são:

\subsection{Objetivo Geral}

Avaliar comparativamente o desempenho de grandes modelos de linguagem na resolução de questões objetivas do ENADE aplicadas a cursos da área de Computação, com base no gabarito oficial.

\subsection{Objetivos Específicos}

Os objetivos específicos são:

\begin{itemize}
    \item Realizar uma análise da literatura recente sobre grandes modelos de linguagem com foco em tarefas de resposta a perguntas e compreensão de texto;

    \item Selecionar, com base na literatura analisada, os modelos de linguagem para fins de avaliação comparativa;
    
    \item Coletar e organizar questões objetivas de múltipla escolha do ENADE, aplicadas desde 2004, para os cursos da área de Computação;
    
    \item Submeter as questões selecionadas aos modelos escolhidos, registrando sistematicamente as respostas geradas;

    \item Avaliar e comparar o desempenho dos modelos considerando a granularidade de curso, a partir dos gabaritos oficiais disponibilizados pelo INEP.
\end{itemize}

\section{Justificativa}
O uso de grandes modelos de linguagem (LLMs) em tarefas de processamento de linguagem natural tem se mostrado promissor em diversos contextos, incluindo aplicações na área educacional. Em especial, na área de Computação, muitos processos avaliativos como a correção de questões de múltipla escolha, que frequentemente envolvem conceitos técnicos  ainda são realizados manualmente ou com sistemas limitados em flexibilidade e capacidade interpretativa. Nesse cenário, os LLMs se destacam por oferecerem maior adaptabilidade e por possibilitarem a geração de feedbacks mais contextualizados e pedagógicos. No entanto, sua real eficácia nesse tipo de aplicação ainda carece de investigações mais aprofundadas. Embora já existam pesquisas voltadas para a análise do desempenho desses modelos em contextos avaliativos, observa-se uma lacuna no que diz respeito ao uso dos LLMs especificamente em questões do ENADE voltadas para cursos da área de Computação. Assim, este trabalho justifica-se pela necessidade de avaliar o desempenho de diferentes LLMs na resolução e interpretação de questões desse exame, buscando compreender suas limitações, potencialidades e possíveis contribuições para o aprimoramento de processos avaliativos na educação superior.

\section{Organização do Trabalho}

Esse trabalho é organizado como segue: No capítulo \ref{fundamentacao} são apresentados os conceitos base para melhor entendimento das tecnologias abordadas. Portanto, o capítulo apresenta uma introdução sobre os grandes modelos de linguagem (Large Language Models), seguido por apresentar o conceito da tecnologia e uma contextualização histórica sobre a evolução da área da PLN até a chegada dos transformers. Ao final do capítulo é também listado as principais pesquisas relacionadas. No capítulo seguinte, \ref{delineamento}, é descrito os passos necessários para realizar o experimento desejado assim como o ambiente adotado para execução da pesquisa. No capítulo \ref{resultados} é apresentado os dados coletados no experimento executado, em seguida é feito uma análise dos dados visando responder as questões de pesquisa. Por fim, no capítulo \ref{conclusoes} é sintetizado o que foi realizado na pesquisa assim como os resultados obtidos ao final da análise de dados.

